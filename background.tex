\section{Background}

Virtualisation in the context of distributed cloud environments usually refers to virtual machines. The core idea is analogous to computer hardware virtualisation. Operating systems offer an interface for the processes to utilize the computer hardware while giving them an illusion that they have all of the hardware for themselves \cite{ArpaciDusseau14-Book}. In reality the resources are shared among many processes. Likewise in cloud environments resources are being share by processes but also by different users running different operating systems, configurations and programs. As with the processes, users are given an impression that they alone have access to the underlying hardware resources, whereas in reality there are multiple users using the same physical machines.
	
			In cloud computing, there are multiple recognized service models which dictate how the users can use the given system and what privileges they are given \cite{Mell:2011:SND:2206223}. In its most limited form, a cloud service is offered to user as a predefined application or a set of applications. The users has some interface for interacting with the applications but is given no control over anything else such as other applications, the operating system the application is running on or network and hardware configurations. This is generally known as Software as a Service or SaaS for short. The most permissible service model is known as IaaS, Infrastructure as a Service. In its archetype the user gets access to all fundamental computing resources, possibly including some network components, and can run arbitrary software including operating systems. The user experience should be similar to that with their personal computers. The user is not allowed to access the underlying cloud infrastructure. 