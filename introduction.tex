\section{Introduction}

Cloud adoption is growing ever so fast with vast majority of both enterprises and small and medium businesses leveraging on cloud computing one way or another. \cite{stateofthecloud} While private cloud usage is growing at steady pace, its growth is eclipsed by that of public cloud usage which is estimated to grow as trice as fast when compared to private clouds. 
Contributing to the accelerated speed of cloud adoption is the trend of simultaneous use of multiple cloud environments and services, both private and public. The concept of using multiple clouds to support and enable same business is called \textit{Hybrid cloud} and on average enterprises report using and experimenting with almost five different clouds simultaneously. 
Another 'bubbling under' trend of cloud computing is a shift away from virtualised clouds to running workloads directly on hardware. This \textit  {bare-metal computing} interests companies running computationally heavy workloads such as Big Data and Machine learning as bare-metal seeks to amend performance overheads inherent to virtualisation. Openstack Foundation report a stark increase in the usage of its bare-metal service Ironic \cite{openstacksurvey} and along with the possibility to use bare-metal servers with major public cloud providers there are also relatively new service providers such as \textit{Vultr}\cite{vultr} and \textit{Packet}\cite{packet} who focus especially on providing bare-metal servers as a service.

Growing usage of both hybrid clouds and the variety of the underlying hardware and interfaces to use them introduce complexity to management of these systems. As a natural reaction there are now many tools to abstract and manage this complexity. For example IBM has their own tool \textit{IBM Multicloud Manager}\cite{ibmmulticloud} and \textit{Rancher} has been a popular framework for handling multiple Kubernetes clusters\cite{Kubernetes}. This thesis focuses on \textit{Cloudify}\cite{cloudify} which is also a
