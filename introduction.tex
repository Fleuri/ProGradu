\section{Introduction}

Cloud adoption is growing ever so fast with vast majority of both enterprises and small and medium businesses leveraging on cloud computing one way or another.\cite{stateofthecloud} While private cloud usage is growing at steady pace, its growth is eclipsed by that of public cloud usage which is estimated to grow trice as fast when compared to private clouds. 
Contributing to the accelerated speed of cloud adoption is the trend of simultaneous use of multiple cloud environments and services, both private and public. The concept of using multiple clouds to support and enable same business is called \textit{Hybrid cloud} and on average enterprises report using and experimenting with almost five different clouds simultaneously. 
Another trend of cloud computing is a shift away from virtualised clouds to running workloads directly on hardware. This \textit  {bare-metal computing} interests companies running computationally heavy workloads such as Big Data and Machine learning as bare-metal seeks to amend performance overheads inherent to virtualisation. Openstack Foundation report a stark increase in the usage of its bare-metal service \textit{Ironic} \cite{openstacksurvey} and along with the possibility to use bare-metal servers with major public cloud providers there are also relatively new service providers such as \textit{Vultr} \cite{vultr} and \textit{Packet} \cite{packet} who focus especially on providing bare-metal servers as a service.

Growing usage of both hybrid clouds and the variety of the underlying hardware and interfaces to use them introduce complexity to management of these systems. As a natural reaction, there are now many tools to abstract and manage this complexity. For example, IBM has their own tool \textit{IBM Multicloud Manager} \cite{ibmmulticloud} and \textit{Rancher} \cite{rancher} has been a popular framework for handling multiple Kubernetes clusters \cite{Kubernetes}. This thesis focuses on \textit{Cloudify} \cite{cloudify} which is also a tool to manage multiple clouds. What sets it apart from others however is the fact that it aims to be a general tool independent of the underlying platform implementations meaning that the user can control multiple clouds and even single physical machines as a generic set of resources without extensive knowledge of their implementation. This opens up avenues in optimising cloud resource usage and introducing hardware that has not traditionally been used as cloud computing resources such as consumer-grade computers and single-board computers such as Raspberry PIs. However, as bare-metal cloud computing is not as popular as applications of virtualised computing resources, Cloudify's bare-metal capabilities remain underdeveloped.

In this thesis I identify shortcomings related to Cloudify's capability of managing generic computational resources such, as consumer-grade computers, and provide prototypical solutions addressing them. Main problems addressed are Cloudify's inability to automatically detect and manage physical hosts in the cluster and its lacking knowledge of the performance capabilities of the said hosts. My key contributions are:

\begin{enumerate}
\item  A software solution which detects joining and parting hosts in the cluster network automatically without a need for human intervention and provides them to the Cloudify Manager for allocation.
\item A modification to Cloudify's Host-pool service so that it retrieves and stores hardware data and performance capabilities of the hosts. In the future Cloudify Manager can use this data to optimise resource usage and make more intelligent workload allocation choices.
\end{enumerate}

Both of the solutions integrate seamlessly with the existing Cloudify components. I also perform experiments on real machines to showcase and validate the capabilities and correctness of my solutions within the scope of this thesis. The features I am addressing are lacking likely because Cloudify's develpment team's focus has been on integrations with the major cloud platforms and generic hardware provisioning is a niche use case compared to them.

The remainder of this thesis is structured as follows: First in section~\ref{background} I give a background overview of common cloud computing concepts. Then I follow with the background review of Cloudify, comparing it conceptually to OpenStack which serves as an example of a typical Cloud computing platform. I also provide a quick overview of hybrid cloud and bare-metal management tools similar to Cloudify.
From section~\ref{systemdesign_and_implementation} onwards I focus on identifying the scope of the prototype and the shortcomings of Cloudify I set out to correct. I provide an overview of the parts in Cloudify with which my proposed system interacts with and detail a high level design of my solutions for automating host detection and retrieval and storage of hardware data. Section~\ref{technical_implementation} presents the lower level details of solutions' implementation followed by the experiments in section~\ref{experiments} showcasing and validating the solutions' capabilities. Finally in chapter~\ref{future} I review future work and research required to fully develop the system beyond the prototype.

Both solutions, Discovery Service and Modified Host-pool Service, presented in this thesis are open source and can be found at \url{https://bitbucket.org/Fleuri/discoveryserviceforcloudify/src/master/}  and \url{https://github.com/Fleuri/cloudify-host-pool-service} respectively.

\pagebreak