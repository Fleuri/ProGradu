\section{Technical Implementation}

% Do these need references?
Discovery Service is implemented with Python programming language and Flask Web framework. They were chosen because all of the components in Cloudify are also written in Python and Flask is used primarily for providing REST APIs. Even though Discovery Service does not provide any REST APIs, Flask is used for configuration management and source code organisation. Naturally, if need arises in the future to expand Discovery Service with a REST API, the development work is streamlined because of the framework.

In addition to Python program, Discovery Service relies on Redis \cite{Redis} as an in-memory key-value storage. Redis is a completely separate process in addition to Discovery Service. The preferred way of deploying Redis is in a docker container as it doesn't require installation or configuration save for exposing a correct port in the container and specifying Redis' address to Discovery Service. Redis could also be installed in the host system or even a remote system, though latter option has no practical purpose due to network latency as Redis achieves its high performance by storing values in the memory instead of disk.

On the source code level, Discovery Service consists of two major components: The network scanner and request service. The network scanner is given a subnet as a parameter and it constantly sniffs the network detecting joining and already present devices and keeping track of them. Its other task is to periodically send health checks to known devices and if a health check fails enough times, it removes the given device from the logical host pool. Request service is responsible for sending HTTP requests to the Cloudify Host-pool service. It is called by the request service and it runs asynchronously. In addition to HTTP requests, it 