\subsection{Virtualisation Techniques}

Traditionally virtualisation has referred to a software abstraction layer residing between the computer hardware and the operating system. \cite{taxonomy} This layer has been called Virtual Machine Monitor (VMM) or more recently a hypervisor and it hides and abstracts the computing resources from the OS, allowing multiple OSs to run simultaneously on the same hardware. There are multiple ways to run hypervisor-based virtualisation. Lately a technology called container-based virtualisation has been gaining popularity. Instead of emulating whole hardware, containers make use of features provided by the host operating system to isolate processes from each other and other containers \cite{eder2016hypervisor}.

\subsubsection{Full virtualisation}

In full virtualisation, the hypervisor runs on top of the host OS. The guest OSs run on top of the hypervisor which in turn emulates the underlying real hardware to them. The guest OSs can be arbitrary. Figure ~\ref{fig:full} shows the full virtualisation architecture with the hypervisor running on top of the Host OS and Guest OSs on top of the hypervisor using their emulated hardware. The main advantage of full virtualisation is that it is easy to deploy and should not pose problems to an average user but the virtualisation overhead results in significantly reduced performance when compared to running directly on hardware. Popular examples of full virtualisation applications are Oracle's \textit{VirtualBox}\cite{VirtualBox} and \textit{VMware Workstation}\cite{WorkStation}. 

\subsubsection{OS-Layer virtualisation}

\subsubsection{Hardware-Layer virtualisation}

\subsubsection{Paravirtualisation}

\subsubsection{Container-based virtualisation}

\begin{figure}
\centering
  \includegraphics[width=8cm,height=5cm]{myPictureName.png}%
  \caption{My Picture}
  \label{fig:full}
\end{figure}